%! TEX program = pdflatex
\documentclass[arbeit=studie,oneside,BCOR=12mm]{ArbeitRST}
\usepackage{amsmath} 
\hypersetup{
    unicode=false,          % non-Latin characters in Acrobat’s bookmarks
    pdftoolbar=true,        % show Acrobat’s toolbar?
    pdfmenubar=true,        % show Acrobat’s menu?
    pdffitwindow=false,     % window fit to page when opened
    pdfstartview={FitH},    % fits the width of the page to the window
    pdftitle={Verbessung der Spurerkennung und -verfolgung autonomer Modellfahrzeuge}, % title
    pdfauthor={James Vero Asghar},     % author
    pdfsubject={Subject},   % subject of the document
    pdfcreator={James Vero Asghar},   % creator of the document
    pdfproducer={Producer}, % producer of the document
    pdfkeywords={Stanley controller} {sliding window method} {model cars}, % list of keywords
    pdfnewwindow=true,      % links in new window
    colorlinks=true,        % false: boxed links; true: colored links
    linkcolor=blue,         % color of internal links (change box color with linkbordercolor)
    citecolor=green,        % color of links to bibliography
    filecolor=magenta,      % color of file links
    urlcolor=cyan           % color of external links
}
\setlength{\parindent}{0ex}
\setlength{\parskip}{2ex}

% Entfernt die farbigen Markierungen - bitte Druckversion mit dieser Option kompilieren
%\hypersetup{hidelinks}

\begin{document}

% Titelseite
% ==========

% Name des Verfassers
\author{James Vero Asghar}

% Geburtsort
\geburtsort{Austin, Texas, Vereignete Staaten}

% Geburtsdatum
\geburtsdatum{3. Dezember 1997}

% Titel der Arbeit
\title{Verbesserung der Spurerkennung und -verfolgung autonomer Modellfahrzeuge}

% Untertitel
\subtitle{}

% Angabe der Betreuer
\betreuer{Dr.-Ing. Carsten Knoll}
\betreuer{M.Sc. Paul Auerbach}

% Datum der Einreichung
\date{2. Februar 2222}


% Zunächst für das Vorgeplänkel römische Seitenzahlen und einfacher Seitenstil
% ============================================================================
\pagenumbering{Roman}
\pagestyle{plain}


% Titelseite erstellen
\maketitle


% Selbstständigkeitserklärung
% ===========================

% Selbstständigkeitserklärung erstellen
\selbststaendigkeitserklaerung


% Kurzfassung / Abstract
% ======================
\kurzfassung{An dieser Stelle fügen Sie bitte eine deutsche Kurzfassung ein.}
{At the Connected Robotics Lab (CoRoLa) research group at the Barkhausen
Institut, a demonstration using remote controlled 2 axis vehicles was
developed. The demonstration is currently in use in order to model
communications between vehicles similar to an Internet of Things (IoT) network.

In order to better model the complexities of a real world system a nonlinear
control algorithm was introduced to the system in order to more accurately
control the vehicles. The Stanley Controller is a nonlinear controller
developed for use with the lateral control of 2 axis vehicles in mind. As input
to the controller, a fisheye camera was used to create images of road. These
images were then digitally processed to find the yellow line for the vehicles
to follow.}


% Inhaltsverzeichnis
% ==================
\tableofcontents

\chapter{Einführung}

\chapter{Regelalgorithmus}

\section{Stanley-Regler}
%The Stanley controller is a non-linear control algorithm developed in 2005 by
%Stanford University. It was developed in order to control a 2 axle vehicle
%using only the heading and the cross track error from the path to be followed.
%It was proven to be asymptotically global stable for the kinematic model of a
%two axle vehicle .However in for the duration of this thesis the the model used
%is not the kinematic model of a 2 axle vehicle instead a model that also
%accounts for the dynamics of the steering angle.
%
%The stanley controller is a path following controller as opposed to a
%trajectory following algorithm. The stanley controller is designed as a lateral
%controller,  so that the vehicle stays on the path, but has no bearing on the
%speed at which it will follow the path. The positives of this approach allow
%for a flexible choice of the speed of the vehicle, which barring dynamic
%effects, allows for a engineer to choose any speed necessary for their
%application.
%
%The stanley controller is represented by the following equation: $$u = \theta_d
%- \theta + \arctan\left(\frac{ke_{fa}}{v}\right),$$ where $u$ is the controller
%output, $\theta$ is the current heading of the vehicle, $\theta_d$ is the
%heading of the path, $k$ is a scaling factor, $v$ is the velocity of the
%vehicle, and $e_{fa}$ is the cross track error from the midpoint of the
%vehicle’s front axle to the path. 
%
%The stanley controller is composed of two components:  a component handling the
%offset of the vehicle from the path and a component handling the derivative of
%the offset (difference in heading).  The first component, represented by
%$\arctan(\frac{ke_{fa}}{v})$,  drives the steering angle to turn more towards
%the path, the farther away the car currently is. The second component,
%represented by $\theta_d - \theta$, controls the steering angle to stay
%parallel with the path to be followed. Working together, the two components
%steer the vehicle on to the path to be followed.

The Stanley controller is a non-linear control algorithm that was developed in
2005 by Stanford University with the purpose of controlling a two-axle vehicle
using only the heading and the cross track error from the path to be followed.
The algorithm has been proven to be asymptotically globally stable for the
kinematic model of a two-axle vehicle. However, it should be noted that the
model used in this thesis accounts for the dynamics of the steering angle
rather than being based solely on the kinematic model of the vehicle.

The Stanley controller is a path following controller rather than a trajectory
following algorithm. As a lateral controller, it is designed to keep the
vehicle on the path but has no impact on the speed at which the vehicle travels
along the path. This approach allows for a flexible choice of the vehicle
speed, which can be selected as required for a given application, subject to
dynamic effects.

The Stanley controller is represented mathematically by the following equation:
\begin{equation}
    u = \theta - \theta_d + \arctan\left(\frac{ke_{fa}}{v}\right),
    \label{eq:Stanley-Regler}
\end{equation}
where $u$ represents the controller output, $\theta$ is the current heading of
the vehicle, $\theta_d$ is the heading of the path, $k$ is a scaling factor,
$v$ is the velocity of the vehicle, and $e_{fa}$ is the cross track error from
the midpoint of the vehicle's front axle to the path.

The Stanley controller consists of two components: a component that handles the
offset of the vehicle from the path and a component that handles the derivative
of the offset (difference in heading). The first component is represented by
$\arctan(\frac{ke_{fa}}{v})$ and drives the steering angle to turn more towards
the path as the car gets farther away from it. The second component,
represented by $\theta_d - \theta$, controls the steering angle to stay
parallel with the path to be followed. Together, these two components work to
steer the vehicle onto the desired path.


\section{Simulation}
In order to investigate the behavior of the Stanley controller on the remote
controlled vehicle, a simulation of the control loop was used. The simulation
has 3 major components: first is the path that the vehicle must follow, second
is the Stanley controller and third is the vehicle model. A differential
equation solver is used in order to analyze the time series of the 2 axle
vehicle. 

The path that the vehicle must follow is a virtual version of the real world
track that the vehicle will use. A visual representation of the path is FIGURE.
In the simulation, the path is represented with the following continuous static
function: ----, with the input t and the output (x, y, h). This function is
then discretized for any chosen amount of t values, in order to create an array
of (x, y, theta) values. This array is then fed into the differential equation
solver that encapsulates the Stanley controller and the vehicle model. 

The vehicle model used is the rear axle vehicle model. This model is
represented by the following nonlinear state space representation: 
\begin{gather}
    \dot{x} = v \cos(\theta) \\
    \dot{y} = v \sin(\theta) \\
    \dot{\theta} = \frac{v}{l}\tan(\varphi) \\
    \dot{\varphi} = \frac{-1}{T}\left(\varphi - \delta\right).
\end{gather}

The state variables of the model are $x_H$, $y_H$, $\phi$ und $\theta$, where
$x_H$ und $y_H$ are the XY-coordinates of the midpoint of the rear axle and
$\theta$ is the heading of the vehicle. $\phi$ is the steering angle of the
vehicle. The input variables of the model are $v$ und $\phi_d$, where $v$ is
the velocity of the vehicle and $\varphi$ is the output of the Stanley
controller. The steering angle $\varphi$ is represented by a first order linear
differential equation with the configurable time constant $T$  The behavior of
the Stanley controller when subjected to dynamic effects on the steering angle,
is of great importance, as the global asymptotic stability of the controller
was only proven on the kinematic model, with a springable steering angle. 

As the Stanley controller requires the midpoint of the front axle for the cross
track error, it must be calculated from the rear axle. The midpoint of the
front axle is calculated from the state variables through the following static
function:
\begin{gather}
    \underline{p}_H := 
    \begin{bmatrix}
        x_H & y_H
    \end{bmatrix}^T \\
    \underline{p}_V := 
    \begin{bmatrix}
        x_V & y_V
    \end{bmatrix}^T
    \label{eq:Hinterradachse und Vorderradachse}
\end{gather}
\begin{equation}
    \underline{p}_V = \underline{p}_H + l 
    \begin{bmatrix}
        \cos(\theta + \pi/2) \\ 
        \sin(\theta + \pi/2),
    \end{bmatrix}
    \label{eq:Transformation von Hinterradachse zu Vorderradachse}
\end{equation}
where xf and yf are the x/y coordinates of this midpoint. 

As seen before, the Stanley controller is composed of multiple parts, which
must be calculated. The heading is a state variable, therefore it is always
available. For the cross track error and the heading path, the correct path
point must be chosen. The Stanley controller uses the closest path point to
determine the required heading and current cross track error. For the
simulation this is determined by finding the point with the smallest distance
from the front axle midpoint. The algorithm is represented as follows: ---.
Through this found point, the heading path is determined. The cross track error
is then determined through the following equation: ---. This equation
represents the dot product of the vector perpendicular to the heading vector
and the vector from the front axle to the path point. FIGURE is a visual
representation of these two vectors. The heading path and cross track error are
then fed into the Stanley controller, whereby the output of the controller is
fed into the vehicle model.

As the Stanley controller only outputs a steering angle, the velocity is
assumed to be constant. A high level diagram of the control loop is represented
in FIGURE.

The Stanley controller works with continuous time signals, which is different
from what is used on the vehicle. The vehicle captures images at a specific
frequency, which are then processed by the pipeline before being fed into the
controller. In order to simulate this behavior, a cache was built into the
simulator, where the heading path and cross track error are saved. The Stanley
controller is then fed these saved values for a chosen amount of time. By
caching these values, the behavior of a zero-order hold is simulated. The
output of this zero-order hold is then fed into the Stanley controller. This
allows for the behavior of the vehicle to be investigated for different
sampling frequencies, which allows for a tolerance to be set for the processing
speed of the pipeline. A comparison of the simulation at various sampling
frequencies is shown in FIGURE.

As can be seen in FIGURE, the Stanley controller is indeed affected by the
sampling frequency, however above X Hz the behavior of the vehicle is not
noticeably different. 

The simulation is then executed for a chosen amount of time. 
    \begin{equation} 
        \underline{x} := 
        \begin{bmatrix}
            x_H & y_H & \theta
        \end{bmatrix}^T 
        \label{Zustandsvektor}
    \end{equation}




    \begin{gather}
        \underline{\nu} := 
        \begin{bmatrix}
            -\cos(\theta + \pi/2) \\
            -\sin(\theta + \pi/2)
        \end{bmatrix} \\
        \underline{p}_P := 
        \begin{bmatrix}
            x_P & y_P
        \end{bmatrix}^T
    \end{gather}

    \begin{equation}
        e_V :=  \underline{\nu} \cdot \min_i |\underline{p}_V -  \underline{p}_{Pi}| ?
        \label{eq:Querabweichung}
    \end{equation}

\chapter{Bildverarbeitung}
\section{Kamera-Kalibrierung}
The start of the image processing pipeline is the camera calibration. In this 
project, a fisheye camera was chosen as opposed to a rectilinear camera. The 
benefit of a 
fisheye camera arises from its wider angle of view in comparison to a 
rectilinear camera. However, a fisheye camera distorts(?) an image differently 
from a rectilinear 
camera. A rectilinear camera preserves straight lines, when no distortion is 
present, as opposed to a fisheye camera, which will always curve straight lines. 
The 
curvature of these straight lines is dependent on their radial distance from 
the center of the image.  An example is in FIGURE.

Camera calibration is used in order to approximate the extrinsic and intrinsic 
parameters of a camera (MATLAB). A calibrated camera allows for 3-D information
to be recovered from a 2-D image. The intrinsic parameters of a camera are often 
represented by the following 3x3 matrix:

$$K = matrix$$, where fx, fy, cx, and cy MEAN SOMETHING. The extrinsic
parameters are often represented by the following row vector:
$$vector$$, where $R$ is the rotation of the camera with respect to a world 
frame and $t$ is the translation from the same world frame. The row 
vector-matrix multiplication of the extrinsics vector and the intrinsics matrix
result in the camera matrix $P$.

A camera is calibrated by using a collection of photos with known straight 
lines. Commonly, a series of checkerboard images with known dimensions are used. 
Photos of the checkerboard are then captured at varying angles and locations in 
the scene. This series of images is then fed to the camera calibration 
algorithm, which will first determine the locations of the checkerboard squares
and the lines connecting them.  Afterwards, using the model of a fisheye camera,
the algorithm 
compensates for the distortion of the fisheye camera. 
For example, as seen in FIGURE, the curved lines of the checkerboard are made 
straight again after calibration. The number of needed images is dependent on 
the specific camera, however a large collection of photos will lead to a more 
accurate parameter approximation. (OPENCV Citation) Using images with a higher 
resolution will also lead to a more accurate parameters. (OPENCV/BLOG Citation)
However, the approximated parameters are only accurate for the calibration of 
images taken at the same resolution as the images used for the calibration. In
order to use the camera matrix for images with a lower resolution, the camera 
must be scaled with the following formula:
(EQUATION). As the camera matrix is an example of an affine transformation, the
value at $C_{3, 3}$ must be reset to 1.
The resulting camera matrix will work for smaller resolutions matching the 
aspect ratio of the images used for calibration, but it has been noticed that 
when the difference in resolution is too large, distortion is added back into 
the image. In FIGURE, the original image is on the left, the middle image is 
calibrated using images at 900p and the right image is calibrated using images 
at 240p. As can be seen, the image on the right straightens the ruler, while 
the middle image leaves it curved.

One downside to camera calibration is that any calibrated image has a lower 
resolution than the original image. This is a consequence of the calibration 
process, as the process will distort a subset of the image, particularly 
involving the pixels around the corners of the image. 
Therefore the calibration process will remove these pixels from the resulting 
image. In order to recreate the image in its original resolution, an 
interpolator is used. 
However, the interpolator will return a blurrier image than the original. 
In order to compensate for this, it is recommended to calibrate a camera with 
high resolution images and then capture images at that resolution. 
Then, instead of using the interpolator, shrink the images down to a resolution
that is necessary for the application.  
This results in a more accurate image without any blurriness. 
Unfortunately, this process is computationally heavy and with the processor 
used on the vehicle, it was decided to only use the images from the interpolator
in order to increase the processing speed.

Another downside from the camera calibration as that the midpoint of the camera
shifts. An example of this can be seen in FIGURE. $C_{3,1}$ and $C_{3,2}$ 
represent the image center, with $C_{3,1}$ being the $x$ coordinate. By manually
changing this value, the midpoint of the image can be shifted back to its 
original location.

\section{Color Thresholding}
After the camera calibration comes the color thresholding stage of the pipeline.
This stage removes all pixels from the image that are not yellow with a high
saturation and brightness. 

At first, the image is filtered of all pixels without a high red component, as 
bright yellow in the red, green and blue (RGB) color space has a high red 
component. 
Afterwards, the image is converted into the Hue, Lightness and Saturation (HLS)
color space. 

Under the RGB color space, yellow is represented as a combination of the red 
and green channels with the blue channel set to 0. Pure yellow is also defined 
as having both the red and green channels equal to one another, therefore 
allowing only 1 degree of freedom to tune the pipeline.  Having only 1 degree 
of freedom leads to issues with tuning, as this reduces the ability for the 
pipeline to account for disturbances, such as differing lighting conditions or 
reflective surfaces.
Under the HLS color space, the hue channel selects the color, the lightness 
channel corresponds to the amount of white or black in the color and the 
saturation channel is a measure of the purity of a hue. (MS ANNO) Once the hue 
is selected, here yellow, the lightness and saturation channels are then used 
to select the specific shade of yellow. Using these 2 channels allows for a 
more robust detection of color.

In order to detect the yellow lane, the pipeline will filter out colors outside 
of the yellow hue range, with a small tolerance, and then truncate the lower 
part of the lightness and saturation channels. As a result of this, only 
relatively pure yellow is left in the image.
The values used in the pipeline for yellow is represented by the following 
range,
TABLE.

An example of the output from this stage of the pipeline is FIGURE.

\section{Perspective Transformation}
The third part of the image processing pipeline is the perspective 
transformation. 
Whenever an image is captured with a camera mounted to the vehicle, the lane 
line will be trapezoidal as opposed to straight and lane seems to end in a 
point at the center of the image. This perspective requires that all 
calculations 
regarding the lane line must compensate for the decreasing width of the line. 
Therefore, to remove this requirement, a perspective transformation is 
employed. Perspective transformation is the process 
by which a subset of an image is sheared and made to fit the entirety of a new 
image. An example is in FIGURE. For this project, the shape of the subset of 
the 
image is selected to be a trapezoid, shown in FIGURE. 
Using this perspective transform, the trapezoidal shape of the lane line is 
corrected 
into one that is straight, shown in FIGURE.

A consequence of this new perspective, is that the resulting 
image is similar to a 2-D plane of the track. The perspective transformation 
simplifies further image processing as well, as all objects outside of the 
trapezoid 
are cropped, leaving only the track. However, as can be seen in FIGURE, the 
perspective transform introduces extra noise into the image. The shearing 
caused by the perspective transform causes pixels from the original image to 
be stretched, thereby polluting the image with noise. 

Therefore, this is the third stage in the pipeline. As already discussed, in 
the second stage of the pipeline, the threshold stage removes the majority of 
unnecessary information from the image, which reduces the amount of noise 
caused by the perspective transform.

\section{Histogram}
After the perspective transformation stage, a histogram of the pixel density 
is created. During this step, the start point for the sliding window method is
found. 
At every column of the image, the number of white pixels are counted. 
The numbers are then placed into a list. The assumption is then that the 
columns containing the largest numbers contain the lane information. The 
index with the number is then selected as the start point for the sliding 
window method. This pipeline stage reduces computation time by filtering out 
unnecessary parts of the image not containing lane information. A visual 
representation of the histogram can be seen in FIGURE.

\section{Gleit-Fenster-Verfahren}
In the fifth stage of the pipeline, the heading and offset of the lane are
calculated through the sliding window method.

First, a chosen number of rectangles (windows) are created. The windows are
aligned in a column and equally sized. Their height is chosen so that the
column spans the entire height of the image, while their width is chosen to be
a factor of the image width. The number of windows is chosen so that the column
entirely or almost entirely  contains the white pixels representing the lane,
as can be seen in FIGURE. If the number of windows is too high, there is a
chance that the method will falsely determine white pixels deep into the image
as a part of the lane. 

Second, the x-coordinate of the midpoint of the first window is set to the peak
of the histogram from the previous stage. Then, the number of white pixels are
then counted inside the window. If the number is above a chosen threshold, the
pixels are added to an array and the mean of the x-coordinates of the contained
pixels is calculated. The x-coordinate of the next window is set to this
calculated mean while being placed on top of the first window. This process is
then repeated for all remaining windows. When no pixels are contained in a
window, the window is ignored and the following window is placed on top of it.
In the image coordinate system, the top left corner of the image is $(0, 0)$,
with the positive x-axis increasing to the right, while the positive y-axis
increases moving down the image. Therefore, the bottom edge of the image would
be any coordinate with a y component equal to the height of the image. Seen
from the engineer's perspective, the windows would be stacked from lowest to
highest, however in the image coordinate space it is inverted. An example can
be seen in FIGURE.

The pixels not located within each window are then filtered out of the image.
The image before and after the filtering process can be seen in FIGURE. Using
the the least squares method, a polynomial is approximated through remaining
pixels.

\section{Berechnung von Ausrichtung und Abstand}
In the final stage of the pipeline, the heading and offset from the lane line
are calculated. 


From the fitted polynomial calculated in the previous stage, an analytical
heading is found at a chosen point along the path with the following equation:
$$\theta = \arctan(f^\prime(y))$$. The inverse tangent of the derivative of
the polynomial returns the heading at a chosen point. For the Stanley
controller, only the difference between the path heading and vehicle heading
is needed, therefore as there is no global frame for the vehicle, the heading
of the vehicle is chosen to be 0!!!!!! at all times. PICTURE WITH VISUAL
REPRSENTATION 

The classic Stanley controller, as described in chapter 2, uses the heading
calculated from the nearest path point. However, as a side effect of the
polynomial fitting, the heading calculated at the bottom of the image is
incorrect. As can be seen in FIGURE, the sliding window method can on average
not detect the lane line close to the bottom of the frame. Due to this, the
fitted polynomial incorrectly approximates the heading. Therefore, the heading
is calculated at a point located within the detected lane line instead.
Moreover, this adds a degree of freedom to the controller. The consequences of
this are discussed in chapter 4.


The offset from the lane line is calculated from the fitted polynomial with
the following function: $$e_{fa} = (\frac{w}{2} - f(h))\cdot x_{mpp},$$ where
$w$ is the width of the image, $f(h)$ is the output of the fitted polynomial
at the bottom of the image, $x_{mpp}$ is a scaling constant to translate
pixels into meters and $e_{fa}$ is the offset from the vehicle to the lane.
All together, the difference of the x-component of the midpoint and the
x-component of the polynomial point located on the bottom edge of the image is
translated into meters.
\section{Pipeline-Optimierungen}
One of the obstacles encountered during the creation of the image processing
pipeline, is that the image would be processed too slowly. 
The initial implementation of the pipeline ran at approximately 15 Hz, which was
decided to be close to the lower bound of our chosen tolerance for the 
processing frequency. 
In order to improve the processing speed of the pipeline to an acceptable level,
only 1 optimization was needed. 
While further optimizations are possible, they were considered to be out of the
scope of the assignment and are therefore not discussed here.

The Python program cProfiler was used in order to tabulate the function calls of
the pipeline, as well as the runtime of each function call. As the pipeline is a
deterministic program, an simple runtime improvement is to cache the output of 
expensive function calls into memory. Therefore during all subsequent function 
calls, the cached result is returned instead of running the calculation again. 
As an example, the perspective transformation matrices are an expensive 
calculation, which when cached led to a speed improvement of the pipeline.

In order to prove the speed improvement of the pipeline, an experiment was 
implemented. First, the pipeline is run through once and the runtime ignored as
the speed improvement is only relevant to subsequent runs of the pipeline. Then,
on the second run of the pipeline, the runtime is measured for both variants, 
with and without caching the expensive function calls from the first run. The 
experiment is then run 100 times and the result tabulated. The arithmetic mean 
of the runtimes for both variants are then compared to one another to measure 
the speed improvement. The values are in TABLE, showing that by caching the 
expensive function calls, the pipeline processing speed improved by VALUE%.

\chapter{Experiment}
\section{Hardware Setup}
The CoRoLa car platform is composed of a Raspberry Pi, a fisheye lens camera, a
DC motor and a motor controller to control the vehicle speed and a servo motor
to control the steering angle of the vehicle. The Raspberry Pi sends PWM
signals to the motor controller and the servo controller in order to control
the vehicle. Using Robot Operating System 2 (ros2), the Raspberry Pi is
connected over the network to a central computer. The Stanley controller runs
on the central computer, which processes the data from the camera and sends the
output of the Stanley controller to the vehicle.

\section{Experiment Setup}
In order to compare the Stanley controller and the sliding window method with
the PID controller, 4 test criteria are chosen: maximum acceptable velocity and
maximum offset angles, as well as the root mean square values of the vehicle
offset and controller outputs.

In order to measure the maximum acceptable velocity, vehicle drives around the
track at a given forward velocity. If the behavior of the vehicle is seen as
"acceptable", then the vehicle is stopped and the velocity is increased. This
process repeats until the behavior of the vehicle is no longer "acceptable".
"Acceptable" behavior of the vehicle is defined empirically under the following
two criteria: ability to follow the lane line and an absence of observed
oscillations.

The measurement of vehicle offset is conducted through the following two
experiments. First, the vehicle starts driving around the track. Then using an
overhead camera, video of the vehicle driving around the curves is recorded.
The vehicle is recorded at differing forward velocities. For the first round of
recordings, the vehicle is controlled by the PID controller and for the second,
it is controlled by the Stanley controller. This experiment is then repeated
with the camera recording video of the vehicle driving along the straights
instead. 

In order to measure the controller output of the two controllers, the vehicle
is commanded to drive around the track at least 3 times. During the drive, the
controller output is recorded with an approximate sampling rate of 20 Hz. After
the drive, the data is graphed over time and a subset of the data is selected.
The subset of the data is selected under the following criteria: 


\chapter{Conclusion}
\end{document}
