%! TEX program = pdflatex
\documentclass[arbeit=studie,oneside,BCOR=12mm]{ArbeitRST}
\usepackage{amsmath} 
\hypersetup{
    unicode=false,          % non-Latin characters in Acrobat’s bookmarks
    pdftoolbar=true,        % show Acrobat’s toolbar?
    pdfmenubar=true,        % show Acrobat’s menu?
    pdffitwindow=false,     % window fit to page when opened
    pdfstartview={FitH},    % fits the width of the page to the window
    pdftitle={Verbessung der Spurerkennung und -verfolgung autonomer Modellfahrzeuge}, % title
    pdfauthor={James Vero Asghar},     % author
    pdfsubject={Subject},   % subject of the document
    pdfcreator={James Vero Asghar},   % creator of the document
    pdfproducer={Producer}, % producer of the document
    pdfkeywords={stanley controller} {sliding window method} {model cars}, % list of keywords
    pdfnewwindow=true,      % links in new window
    colorlinks=true,        % false: boxed links; true: colored links
    linkcolor=blue,         % color of internal links (change box color with linkbordercolor)
    citecolor=green,        % color of links to bibliography
    filecolor=magenta,      % color of file links
    urlcolor=cyan           % color of external links
}
\setlength{\parindent}{0ex}
\setlength{\parskip}{2ex}

% Entfernt die farbigen Markierungen - bitte Druckversion mit dieser Option kompilieren
%\hypersetup{hidelinks}

\begin{document}

% Titelseite
% ==========

% Name des Verfassers
\author{James Vero Asghar}

% Geburtsort
\geburtsort{Austin, Texas, Vereignete Staaten}

% Geburtsdatum
\geburtsdatum{3. Dezember 1997}

% Titel der Arbeit
\title{Verbesserung der Spurerkennung und -verfolgung autonomer Modellfahrzeuge}

% Untertitel
\subtitle{}

% Angabe der Betreuer
\betreuer{Dr.-Ing. Carsten Knoll}
\betreuer{M.Sc. Paul Auerbach}

% Datum der Einreichung
\date{2. Februar 2222}


% Zunächst für das Vorgeplänkel römische Seitenzahlen und einfacher Seitenstil
% ============================================================================
\pagenumbering{Roman}
\pagestyle{plain}


% Titelseite erstellen
\maketitle


% Selbstständigkeitserklärung
% ===========================

% Selbstständigkeitserklärung erstellen
\selbststaendigkeitserklaerung


% Kurzfassung / Abstract
% ======================
\kurzfassung{An dieser Stelle fügen Sie bitte eine deutsche Kurzfassung ein.}
{At the Connected Robotics Lab (CoRoLa) research group at the Barkhausen
Institut, a demonstration using remote controlled 2 axis vehicles was
developed. The demonstration is currently in use in order to model
communications between vehicles similar to an Internet of Things (IoT) network.

In order to better model the complexities of a real world system a nonlinear
control algorithm was introduced to the system in order to more accurately
control the vehicles. The Stanley Controller is a nonlinear controller
developed for use with the lateral control of 2 axis vehicles in mind. As input
to the controller, a fisheye camera was used to create images of road. These
images were then digitally processed to find the yellow line for the vehicles
to follow.}


% Inhaltsverzeichnis
% ==================
\tableofcontents

\chapter{Einführung}

\chapter{Regelalgorithmus}

\section{Stanley-Regler}


Der Stanley-Regler ist ein nichtlinearer Regelungsalgorithmus, der 2005 von
der Stanford University entwickelt wurde, um ein zweiachsiges Fahrzeug nur
anhand der Ausrichtung und der Querabweichung des zu verfolgenden Kurses zu
steuern. Der Algorithmus hat sich für das kinematische Modell eines
zweiachsigen Fahrzeugs als asymptotisch global stabil erwiesen.

Der Stanley-Regler ist ein Kursfolger anstelle eines Trajektorienfolgers. Als
Querregler (?) soll er das Fahrzeug auf dem Kurs halten, hat aber keinen
Einfluss auf die Geschwindigkeit, mit der sich das Fahrzeug auf dem Kurs
bewegt. Dieser Ansatz ermöglicht eine flexible Wahl der
Fahrzeuggeschwindigkeit, die unter Berücksichtigung dynamischer Effekte nach
Bedarf für eine bestimmte Anwendung gewählt werden kann. (?)

Der Stanley-Regler wird mathematisch durch die folgende Gleichung dargestellt:
\begin{equation} 
  u = \theta - \theta_d + \arctan\left(\frac{ke_{fa}}{v}\right),
  \label{eq:Stanley-Regler} 
\end{equation}
wobei $u$ den Reglerausgang, $\theta$ die aktuelle Ausrichtung des Fahrzeugs,
$\theta_d$ die Pfadausrichtung, $k$ einen Skalierungsfaktor, $v$ die
Geschwindigkeit des Fahrzeugs und $e_{fa}$ die Querabweichung vom Mittelpunkt
der Vorderachse des Fahrzeugs zum Kurs darstellt.

Der Stanley-Regler besteht aus zwei Komponenten: einer Komponente, die die
Abweichung des Fahrzeugs vom Kurs behandelt, und einer Komponente, die die
Differenz zwischen der Ausrichtung des Fahrzeugs und der des Kurses behandelt.
Die erste Komponente wird durch $\arctan(\frac{ke_{fa}}{v})$ dargestellt und
steuert den Lenkwinkel so, dass er sich auf den Kurs zubewegt, wenn sich das
Fahrzeug weiter von ihm entfernt. Die zweite Komponente, dargestellt durch
$\theta_d - \theta$, steuert den Lenkwinkel so, dass er parallel zum zu
verfolgenden Kurs bleibt. Zusammen bewirken diese beiden Komponenten, dass das
Fahrzeug auf den gewünschten Kurs gelenkt wird.


\subsection{Kleinsignalverhalten}


Um das Verhalten des Stanley-Reglers besser zu verstehen, wird der Regler um
den Punkt \(\links(x_V, \theta_d\rechts)\)= \(\links(0, 0\rechts)\)
linearisiert, was zu folgender Gleichung führt: 
\begin{equation} \bar{u}
    \approx \bar{\theta} + \frac{k}{v}e_{fa}. 
    \label{eq:linearer Stanley-Regler}
\end{equation}

Die sich daraus ergebende Gleichung \eqref{eq:linearer Stanley-Regler} hat die
Form eines PID-Reglers (Proportional-, Integral- und Derivativregler), der nur
P- und D-Komponenten hat. Die P-Komponente wird durch \(\frac{ke_{fa}}{v}\)
dargestellt. Da die Vorwärtsgeschwindigkeit des Fahrzeugs als konstant
angenommen wird, besteht eine Proportionalität zwischen den Ableitungen von
\(e_{fa}\) nach Zeit und Raum. Die Ableitung von \(e_{fa}\) nach dem Raum ist
\(\theta\). Daher ist \(\theta\) die D-Komponente des linearisierten
Stanley-Reglers.

Folglich ist das Verhalten des Stanley-Reglers für kleine Eingangssignale
ähnlich wie das eines PD-Reglers mit dem Fehlerterm \(e_{fa}\). 


\subsection{Gro{\ss}signalverhalten}

Bei großen Signalen wird der Stanley-Regler durch das Verhalten der
\(arctan(...)\)-Funktion sowie durch die zyklische Natur der Ausrichtung
\(\theta\) dominiert. Die Funktion \(arctan(...)\) ist auf einen Wert zwischen
-90 und 90 Grad begrenzt und ist glatt. Die Ausrichtung ist ebenfalls auf den
vorderen Halbkreis bzw. -90 und 90 Grad begrenzt, da davon ausgegangen wird,
dass sich das Fahrzeug in eine Richtung bewegt. (?) Zusammen begrenzen diese
Komponenten den Lenkwinkel (?) des Fahrzeugs.


%Beim Umgang mit großen Signalen wird das Verhalten des Stanley-Reglers
%hauptsächlich von zwei Faktoren beeinflusst: der zyklischen Natur der
%Ausrichtung \(\theta\) und der \(arctan(...)\)-Funktion. Die Funktion
%\(arctan(...)\) ist eine glatte Funktion, die auf einen Bereich zwischen -90
%und 90 Grad begrenzt ist. Andererseits ist die Ausrichtung auf den vorderen
%Halbkreis beschränkt, der zwischen -90 und 90 Grad liegt, da davon ausgegangen
%wird, dass sich das Fahrzeug in eine Richtung bewegt. Diese beiden Faktoren
%wirken also zusammen, um den Lenkeinschlag des Fahrzeugs zu begrenzen und zu
%verhindern, dass er zu groß wird.

\section{Simulation}


Um ein besseres Verständnis für das Verhalten des Stanley-Reglers in einem
ferngesteuerten Fahrzeug zu gewinnen, haben wir eine Simulation des
Regelkreises durchgeführt. Die Simulation umfasst drei Hauptkomponenten: den
Kurs, dem das Fahrzeug folgen muss, den Stanley-Regler und das Fahrzeugmodell.
Zur Untersuchung der Zeitreihendaten des zweiachsigen Fahrzeugs in der
Simulation wurde ein Differentialgleichungslöser eingesetzt.

Der Kurs, dem das Fahrzeug folgen muss, ist eine virtuelle Darstellung der
realen Strecke (?), die das Fahrzeug durchfahren wird. (?) Eine visuelle
Darstellung des Kurses findet sich in ABBILDUNG. In der Simulation wird der
Kurs durch die folgende kontinuierliche statische Funktion definiert:
\begin{equation} 
  (x, y, h) = f(t, t_f), 
\end{equation} 
wobei $t$ und $t_f$ die Eingangsparameter Zeit bzw. Simulationsendzeit (?)sind
und $(x, y, h)$ die kartesischen Koordinaten des Pfadpunkts und die
entsprechende Ausrichtung an diesem Punkt darstellt. Diese Funktion wird für
eine beliebige Anzahl von $t$-Werten diskretisiert, um ein Array von $(x, y,
theta)$-Werten zu erzeugen. Der Differentialgleichungslöser bezieht den
Stanley-Controller und das Fahrzeugmodell ein, um die Zeitreihen des
zweiachsigen Fahrzeugs unter Verwendung dieses Arrays als Eingabe zu
analysieren. (?)


Das verwendete Fahrzeugmodell ist das Modell eines Hinterachsfahrzeugs. Dieses
Modell wird durch die folgende nichtlineare Zustandsraumdarstellung
dargestellt: 
\begin{gather} 
  \dot{x} = v \cos(\theta) \\ 
  \dot{y} = v \sin(\theta) \\ 
  \dot{\theta} = \frac{v}{l}\tan(\varphi) \\
  \dot{\varphi} = \frac{\left(\delta\right - u)}{T}. 
\end{gather}

Die Zustandsvariablen des Modells sind $x_H$, $y_H$, $\phi$ und $\theta$, wobei
$x_H$ und $y_H$ die kartesischen Koordinaten des Mittelpunkts der Hinterachse
sind und $\theta$ die Ausrichtung des Fahrzeugs ist. $\varphi$ ist der
Lenkwinkel des Fahrzeugs. Die Eingangsgrößen des Modells sind $v$ und
$\varphi$, wobei $v$ die Geschwindigkeit des Fahrzeugs und $\varphi$ der
Lenkwinkel ist. Der Lenkwinkel $\varphi$ ist selbst ein dynamisches System, das
durch eine lineare Differentialgleichung erster Ordnung mit der
konfigurierbaren Zeitkonstante $T$ dargestellt wird. Das Verhalten des
Stanley-Reglers bei dynamischen Einflüssen auf den Lenkwinkel ist von großer
Bedeutung, da die globale asymptotische Stabilität des Reglers nur für das
kinematische Zweiachsmodell nachgewiesen wurde.

Da der Stanley-Regler für die Querabweichung den Mittelpunkt der Vorderachse
benötigt, muss dieser aus der Hinterachse berechnet werden. Der Mittelpunkt der
Vorderachse wird aus den Zustandsvariablen durch die folgende statische
Funktion berechnet: 
\begin{gather}
  \underline{p}_H := 
  \begin{bmatrix}
    x_H & y_H
  \end{bmatrix}^T \\
  \underline{p}_V := 
  \begin{bmatrix}
    x_V & y_V
  \end{bmatrix}^T
  \label{eq:Rear Axle and Front Axle}
\end{gather}
\begin{equation}
  \underline{p}_V = \underline{p}_H + l 
  \begin{bmatrix}
    \cos(\theta + \pi/2) \\ 
    \sin(\theta + \pi/2),
  \end{bmatrix}
  \label{eq:Transformation from Rear Axle to Front Axle}
\end{equation}
wobei $x_V$ und $y_V$ die kartesischen Koordinaten dieses Mittelpunkts sind. 

Wie bereits erwähnt,(?) besteht der Stanley-Regler aus mehreren Teilen, die
getrennt voneinander berechnet werden müssen. Die Ausrichtung $\theta$ ist eine
Zustandsvariable, daher ist sie immer verfügbar. Um die Querabweichung und die
Kursrichtung (?) zu berechnen, muss der richtige Bahnpunkt gewählt werden. Der
Stanley-Regler verwendet den nächstliegenden Bahnpunkt vom
Vorderachsmittelpunkt des Fahrzeugs, um die Ausrichtung des Kurses $\theta_d$
und die aktuelle Querabweichung $e_{fa}$ zu bestimmen. Für die Simulation wird
dies bestimmt, indem der Punkt mit dem geringsten Abstand zum
Vorderachsmittelpunkt $\delta\unterline{x}$ gefunden wird. An diesem Punkt wird
dann die Ausrichtung des Kurses ermittelt. Das Skalarprodukt zwischen dem
Vektor senkrecht zur Fahrzeugausrichtung $\underline{x}_{\perp}$ und dem Vektor
vom Pfadpunkt zur Fahrzeugvorderachse $\delta\underline{x}$ ist die
Querabweichung $e_{fa}$. Zur Verdeutlichung ist in ABBILDUNG eine visuelle
Darstellung dieser beiden Vektoren und ihres Skalarprodukts abgebildet. Dieser
Algorithmus wird in ALGORITHM dargestellt. 

Der Kurs und die Querabweichung werden dann in den Stanley-Regler eingegeben,
und der Ausgang des Reglers wird in das Fahrzeugmodell eingespeist. Ein
Übersichtsdiagramm des Regelkreises ist in ABBILDUNG dargestellt.


Der Stanley-Regler arbeitet mit zeitkontinuierlichen Signalen, die sich von den
vom Fahrzeug verwendeten zeitdiskreten Signalen unterscheiden. Das Fahrzeug
nimmt Bilder mit einer festen Abtastfrequenz auf, die von der Pipeline
verarbeitet werden, bevor sie in den Regler eingespeist werden. Um dieses
Verhalten zu simulieren, wird ein Cache in den Simulator eingebaut, um die
Ausrichtung und die Querabweichung zu speichern. Der Stanley-Regler verwendet
dann diese zwischengespeicherten Werte für eine bestimmte Dauer. Dieser
Cache-Ansatz ahmt das Verhalten eines Halteglieds nullter Ordnung nach, und die
Ausgabe wird anschließend in den Stanley-Regler eingespeist. Dieser
Simulationsansatz ermöglicht die Untersuchung des Fahrzeugverhaltens für
verschiedene Abtastfrequenzen und erlaubt die Festlegung einer Toleranz für die
Rechengeschwindigkeit (?) der Pipeline. Der Vergleich der Simulationsergebnisse bei
verschiedenen Abtastfrequenzen ist in ABBILDUNG dargestellt.

Wie in ABBILDUNG dargestellt, wird der Stanley-Regler durch die Abtastfrequenz
beeinflusst, wobei bei niedrigeren Frequenzen ein deutlicher Effekt zu
beobachten ist. Oberhalb eines bestimmten Schwellenwerts von X Hz scheint das
Verhalten des Fahrzeugs jedoch nicht wesentlich durch die Abtastfrequenz
beeinflusst zu werden. (?)

Die Simulation wird dann für eine ausgewählte Zeitspanne durchgeführt. 



\chapter{Bildverarbeitung}
\section{Kamera-Kalibrierung}




\section{Color Thresholding}

\section{Perspectivtransformation}

\section{Histogramm}

Nach der Perspektivtransformation wird ein Histogramm der Pixelanzahl erstellt,
um den Startpunkt für die Gleitfenstermethode zu bestimmen. Die Anzahl der
weißen Pixel wird für jede Spalte des Bildes gezählt, und diese Zahlen werden
in einer Liste gespeichert. Es wird davon ausgegangen, dass die Spalten mit den
höchsten Zahlen Fahrspurinformationen enthalten, und der Index der Spalte mit
der höchsten Zahl wird als Startpunkt für die Gleitfenstermethode gewählt.
Dieser Schritt trägt dazu bei, die Berechnungszeit zu verkürzen, indem Teile
des Bildes herausgefiltert werden, die keine Fahrspurinformationen enthalten.
Eine visuelle Darstellung des Histogramms ist in ABBILDUNG zu sehen.

\section{Gleitfensterverfahren}
\section{Berechnung von Ausrichtung und Abstand}
\section{Ausrei{\ss}er}
\section{Pipeline-Optimierungen}

\chapter{Experiment}
\section{Hardware Setup}

Die CoRoLa-Car-Platform besteht aus einem Raspberry Pi, einer Fisheye-Kamera,
einem Gleichstrommotor und einem Motorregler zur Steuerung der
Fahrzeuggeschwindigkeit und einem Servomotor um den Lenkeinschlag des Fahrzeugs
zu steuern. Der Raspberry Pi sendet PWM Signale an den Motorregler und den
Servoregler, um das Fahrzeug zu steuern. das Fahrzeug zu steuern. Mit dem Robot
Operating System 2 (ros2) wird der Raspberry Pi über das Netzwerk mit einem
zentralen Computer verbunden. Der Stanley-Controller läuft auf dem
Zentralrechner, der die Daten von der Kamera verarbeitet und die Ausgabe des
Stanley-Reglers an das Fahrzeug sendet.

\section{Experiment Setup}


\chapter{Conclusion}
\end{document}
